\documentclass[10pt,a4paper]{article}
\bibliographystyle{ACM-Reference-Format}
\usepackage[colorlinks=true, urlcolor=blue, linkcolor=red]{hyperref}
\hypersetup{colorlinks=true}
\usepackage{graphicx}
\graphicspath{ {./images/} }

\title{}
\author{Antonio Facchiano\\Salvatore Ruocco\\Simone Vittoria}
\date{10 Gennaio 2024}

\renewcommand\contentsname{Indice}

\begin{document}
	\begin{figure}
		\centering
		\includegraphics[scale=0.5]{icon}
	\end{figure}
	
	\maketitle
	\newpage
	{
		\hypersetup{linkcolor=black}
		\tableofcontents
	}
	\newpage
	\section{Introduzione}
	\subsection{Ruzzle}
	Ruzzle è un videogioco sviluppato da \href{https://www.maginteractive.com}{MAG Interactive}, rilasciato nel Marzo del 2012 sugli store di Android e iOS.\\
	Il meccanismo di gioco è ispirato ai giochi da tavolo \textit{Il Paroliere} e \textit{Scarabeo}.\\
	Ciascuna partita è divisa in tre round, e il punteggio finale è dato dalla somma dei punteggi ottenuti nei singoli round. In ciascun round il giocatore ha due minuti a disposizione per formare il maggior numero di parole di senso compiuto con le sedici lettere a disposizione nella \textbf{griglia 4x4} sullo schermo. Le parole devono essere di almeno 2 lettere e devono essere formate unendo lettere adiacenti fra loro in orizzontale, verticale o diagonale. Non è possibile inserire la stessa casella-lettera più volte all'interno della stessa parola.\\
	Come nello \textit{Scarabeo}, a ciascuna lettera è assegnato un punteggio in base alla difficoltà di inserirla all'interno di parole di senso compiuto; ad esempio vocali comuni come A, E, I, O valgono 1 punto, mentre le consonanti più rare come la Z o la H valgono 8 punti.\\
	Il punteggio totale assegnato a ciascuna parola trovata è dato dalla somma dei punteggi assegnati alle singole lettere più un "bonus lunghezza" per le parole più lunghe di cinque lettere. È possibile aumentare il proprio punteggio utilizzando le lettere contrassegnate da simboli-bonus: DL duplica il valore relativo alla lettera, TL triplica il suo valore; DW duplica e TW triplica il valore totale della parola. Il numero delle lettere bonus varia a seconda del round: nel primo round sono presenti solamente una DL e una TL, nel secondo round compare anche una DW mentre nel terzo sono presenti anche due DW e una TW. 
	\subsection{Obiettivi}
	Lo scopo di questo progetto è quello di creare un'IA capace di trovare tutte le parole italiane di senso compiuto contenute in una griglia 4x4 di caratteri data in input.
	Quindi restituire in output le parole trovate e le coordinate all'interno della griglia.\\
	Al momento, abbiamo deciso di non dare un punteggio alle parole trovate.
	\subsection{Specifica PEAS}
	Un ambiente viene generalmente descritto tramite la specifica PEAS, ovvero \textbf{P}erformance, \textbf{E}nvironment, \textbf{A}ctuators, \textbf{S}ensors.
	\begin{itemize}
		\item \textbf{P}: sono le misure di prestazione adottate per valutare l’operato di un agente, in questo caso vogliamo che l'agente sia completo e veloce.
		\item \textbf{E}: descrizione degli elementi dell'ambiente. Il nostro ambiente è una griglia 4x4 dove ogni casella è un carattere dell'alfabeto italiano.
		\item \textbf{A}: gli attuatori a disposizione dell'agente per intraprendere le azioni. In questo caso i nostri attuatori sono le otto direzioni: nord, nord-est, est, sud-est, sud, sud-ovest, ovest, nord-ovest.
		\item \textbf{S}: i sensori attraverso i quali riceve gli input percettivi.
	\end{itemize}
	\subsection{Caratteristiche dell'ambiente}
	\begin{itemize}
		\item \textbf{Parzialmente osservabile}: l'agente non conosce a priori le coordinate di un carattere, ma data una coordinata, può conoscere il carattere contenuto in quel punto della griglia.
		\item \textbf{Deterministico}: lo stato successivo dell’ambiente è completamente determinato dallo stato corrente e dall’azione eseguita dall’agente.
		\item \textbf{Episodico}: l’esperienza dell’agente è divisa in “episodi” atomici, dove ciascun episodio consiste nell’eseguire una singola azione.
		\item \textbf{Statico}: l'ambiente non muta durante l'esecuzione dell'agente.
		\item \textbf{Discreto}: l’ambiente fornisce un numero limitato di percezioni e azioni distinte, chiaramente definite.
		\item \textbf{Singolo agente}: l'ambiente consente la presenza di  un singolo agente con lo scopo di trovare tutte le parole di senso compiuto possibili.
	\end{itemize}
	\subsection{Formulazione del problema}
	\begin{itemize}
		\item \textbf{Stato iniziale}: griglia 4x4 dove ogni casella contiene una lettere dell'alfabeto italiano.
		\item \textbf{Azioni}: otto direzioni: nord, nord-est, est, sud-est, sud, sud-ovest, ovest, nord-ovest. Queste non sono sempre tutte possibili nel caso in cui ci troviamo ai limiti della griglia, oppure la casella corrispondente è stata già visitata.
		\item \textbf{Modello di transizioni}: ad ogni azione si andrà a controllare se il carattere contenuta nella casella appena visitata è utile per costruire una parola di senso compiuto.
		\item \textbf{Test obiettivo}: trovare tutte le parole di senso compiuto possibili.
		\item\textbf{Costo di cammino}: ogni azione ha lo stesso costo.
	\end{itemize}
	\newpage
	\section{Algoritmi di ricerca}
	
\end{document}